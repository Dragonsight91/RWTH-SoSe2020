% general format definition
\documentclass[a4paper, 11pt]{article}
\usepackage[margin=.9in]{geometry}
\usepackage[utf8]{inputenc}
\usepackage[T1]{fontenc}
\usepackage[english]{babel}

% extra packages
\usepackage{hyperref}
\usepackage{listings}
\usepackage{color}
\usepackage{amssymb}
\usepackage{amsmath}
\usepackage{mathtools}
\usepackage{microtype}
\usepackage{stmaryrd}
\usepackage{tikz}
\usepackage{booktabs}
\usepackage{stmaryrd}

% special math symbols
\newcommand{\R}{\ensuremath{\mathbb{R}}}
\newcommand{\N}{\ensuremath{\mathbb{N}}}
\newcommand{\Z}{\ensuremath{\mathbb{Z}}}
\newcommand{\Q}{\ensuremath{\mathbb{Q}}}

% Fraktur für Strukturen
\newcommand{\A}{\ensuremath{\mathfrak A}}
\newcommand{\B}{\ensuremath{\mathfrak B}}
\newcommand{\C}{\ensuremath{\mathfrak C}}
\newcommand{\I}{\ensuremath{\mathfrak I}}

% logical operators
\newcommand{\xor}{\ensuremath{\oplus}} %exklusives oder
\newcommand{\impl}{\ensuremath{\rightarrow}} %logische Implikation

% nicer symbols
\renewcommand{\phi}{\varphi}
\renewcommand{\theta}{\vartheta}
\renewcommand{\epsilon}{\varepsilon}

% Syntax highlighting
\definecolor{commentsColor}{rgb}{0.497495, 0.497587, 0.497464}
\definecolor{keywordsColor}{rgb}{0.000000, 0.000000, 0.635294}
\definecolor{stringColor}{rgb}{0.558215, 0.000000, 0.135316}
\lstset{
  backgroundcolor=\color{white},                        % choose the background color; you must add \usepackage{color} or \usepackage{xcolor}
  basicstyle=\footnotesize,                             % the size of the fonts that are used for the code
  breakatwhitespace=false,                              % sets if automatic breaks should only happen at whitespace
  breaklines=true,                                      % sets automatic line breaking
  captionpos=b,                                         % sets the caption-position to bottom
  commentstyle=\color{commentsColor}\textit,            % comment style
  deletekeywords={},                                    % if you want to delete keywords from the given language
  escapeinside={\%*}{*)},                               % if you want to add LaTeX within your code
  extendedchars=true,                                   % lets you use non-ASCII characters; for 8-bits encodings only, does not work with UTF-8
  frame=tb,	                   	                        % adds a frame around the code
  keepspaces=true,                                      % keeps spaces in text, useful for keeping indentation of code (possibly needs columns=flexible)
  keywordstyle=\color{keywordsColor}\bfseries,          % keyword style
  otherkeywords={True,False,true,false,null,None,NULL}, % if you want to add more keywords to the set
  numbers=left,                                         % where to put the line-numbers; possible values are (none, left, right)
  numbersep=5pt,                                        % how far the line-numbers are from the code
  numberstyle=\tiny\color{commentsColor},               % the style that is used for the line-numbers
  rulecolor=\color{black},                              % if not set, the frame-color may be changed on line-breaks within not-black text (e.g. comments (green here))
  showspaces=false,                                     % show spaces everywhere adding particular underscores; it overrides 'showstringspaces'
  showstringspaces=false,                               % underline spaces within strings only
  showtabs=false,                                       % show tabs within strings adding particular underscores
  stepnumber=1,                                         % the step between two line-numbers. If it's 1, each line will be numbered
  stringstyle=\color{stringColor},                      % string literal style
  tabsize=2,	                                          % sets default tabsize to 2 spaces
  title=\lstname,                                       % show the filename of files included with \lstinputlisting; also try caption instead of title
  columns=fixed,                                        % Using fixed column width (for e.g. nice alignment)
}

\author{Emma van Emelen}
\title{FoSAP Lecture Notes}
\begin{document}
    % titlepage
    \maketitle
    \newpage

    % contents
    \tableofcontents
    \newpage

    % section 1
    \section{Structure and Tasks}
    \subsection{The structure of computer systems}
    \subsection{The Structure and Tasks of an OS}
    \subsection{Multitasking \& Interrupts}

    \section{C Programming}
    \section{Shell Programming}
    % section 4
    \section{Process Management}
    \subsection{The Definition of a Process}
    A process is simply a running program. This program consists of sequentially executed code, which defines state changes.
    Part of a process is a full description of the CPUs current state, like the program counter, registers, the stack, 
    the heap and any variables or filesthat are being worked on\\
    A process is created like this:
    \begin{enumerate}
      \item Create the Process evironment in RAM.
      \begin{itemize}
        \item create the heap
        \item create the stack
        \item load program code
        \item load libraries
      \end{itemize} 
      \item wait for CPU assignment
      \item Load operation into CPU register and execute
      \item If a process is interrupted, its state is saved and another process will get CPU time.
    \end{enumerate}
    \subsection{Memory Allocation}
    A process gets 3GiB of memory by default. Memory allocation is broken up into 5 parts.
    \begin{itemize}
      \item The stack, which holds function parameters and local variables. This is automatically managed memory.
      \item The heap, which is dynamically allocated memory \texttt{malloc}. This gives the programmer full control over the Memory management
            but can cause alot of problems, if not used carefully. Heap-Buffer-Overflows are very common when manually controlling the memory
            allocation. This happens when a program writes more data into memory than the variable has been assigned. This then overwrites any 
            memory that comes after the variable's memory space.
      \item Statically allocated memory and global data. this is broken up into bss, for \lstinline{NULL-}initialized variables and data,
            for other variables.
      \item The code, which holde the executable program code
    \end{itemize}

    \subsection{Process Types}
    \begin{enumerate}
      \item \texttt{The User Process}, for System Programs and Applications. These all run in \texttt{User-Mode}
      \item \texttt{The System Process}, forparts of the Kernel. This includes but is not limited
            to memory management, file system management and device management. These Processes have a higher
            priority and operate in \texttt{Kernel-Mode}.
    \end{enumerate}
    There are multiple active processes at any given time, which have to be coordinated efficiently, because they can all
    use the same resources. These recources may include memory space, CPU time and IO-devices, such as storage, sensors, screens and peripherals.

    \subsubsection{Process Coordination}

    \subsubsection{Pipes}
    \subsubsection{Message Passing}
    \subsubsection{Shared Memory}

    \subsection{Threads}


\end{document}

