% general format definition
\documentclass[a4paper, 11pt]{article}
\usepackage[margin=.9in]{geometry}
\usepackage[utf8]{inputenc}
\usepackage[T1]{fontenc}
\usepackage[english]{babel}

% extra packages
\usepackage{hyperref}
\usepackage{listings}
\usepackage{color}
\usepackage{amssymb}
\usepackage{amsmath}
\usepackage{mathtools}
\usepackage{microtype}
\usepackage{stmaryrd}
\usepackage{tikz}
\usepackage{booktabs}
\usepackage{stmaryrd}
\usepackage{enumitem}

% special math symbols
\newcommand{\R}{\ensuremath{\mathbb{R}}}
\newcommand{\N}{\ensuremath{\mathbb{N}}}
\newcommand{\Z}{\ensuremath{\mathbb{Z}}}
\newcommand{\Q}{\ensuremath{\mathbb{Q}}}

% Fraktur für Strukturen
\newcommand{\A}{\ensuremath{\mathfrak A}}
\newcommand{\B}{\ensuremath{\mathfrak B}}
\newcommand{\C}{\ensuremath{\mathfrak C}}
\newcommand{\I}{\ensuremath{\mathfrak I}}

% logical operators
\newcommand{\xor}{\ensuremath{\oplus}} %exklusives oder
\newcommand{\impl}{\ensuremath{\rightarrow}} %logische Implikation

% nicer symbols
\renewcommand{\phi}{\varphi}
\renewcommand{\theta}{\vartheta}
\renewcommand{\epsilon}{\varepsilon}

% Syntax highlighting
\definecolor{commentsColor}{rgb}{0.497495, 0.497587, 0.497464}
\definecolor{keywordsColor}{rgb}{0.000000, 0.000000, 0.635294}
\definecolor{stringColor}{rgb}{0.558215, 0.000000, 0.135316}
\definecolor{brightlavender}{rgb}{0.75, 0.58, 0.89}
\definecolor{brilliantrose}{rgb}{1.0, 0.33, 0.64}
\definecolor{canaryyellow}{rgb}{1.0, 0.94, 0.0}
\definecolor{cyan}{rgb}{0.0, 1.0, 1.0}
\definecolor{fulvous}{rgb}{0.86, 0.52, 0.0}
\definecolor{olive}{rgb}{0.5, 0.5, 0.0}
\lstset{
  backgroundcolor=\color{white},                        % choose the background color; you must add \usepackage{color} or \usepackage{xcolor}
  basicstyle=\footnotesize,                             % the size of the fonts that are used for the code
  breakatwhitespace=false,                              % sets if automatic breaks should only happen at whitespace
  breaklines=true,                                      % sets automatic line breaking
  captionpos=b,                                         % sets the caption-position to bottom
  commentstyle=\color{commentsColor}\textit,            % comment style
  deletekeywords={},                                    % if you want to delete keywords from the given language
  escapeinside={\%*}{*)},                               % if you want to add LaTeX within your code
  extendedchars=true,                                   % lets you use non-ASCII characters; for 8-bits encodings only, does not work with UTF-8
  frame=tb,	                   	                        % adds a frame around the code
  keepspaces=true,                                      % keeps spaces in text, useful for keeping indentation of code (possibly needs columns=flexible)
  keywordstyle=\color{keywordsColor}\bfseries,          % keyword style
  otherkeywords={True,False,true,false,null,None,NULL}, % if you want to add more keywords to the set
  numbers=left,                                         % where to put the line-numbers; possible values are (none, left, right)
  numbersep=5pt,                                        % how far the line-numbers are from the code
  numberstyle=\tiny\color{commentsColor},               % the style that is used for the line-numbers
  rulecolor=\color{black},                              % if not set, the frame-color may be changed on line-breaks within not-black text (e.g. comments (green here))
  showspaces=false,                                     % show spaces everywhere adding particular underscores; it overrides 'showstringspaces'
  showstringspaces=false,                               % underline spaces within strings only
  showtabs=false,                                       % show tabs within strings adding particular underscores
  stepnumber=1,                                         % the step between two line-numbers. If it's 1, each line will be numbered
  stringstyle=\color{stringColor},                      % string literal style
  tabsize=2,	                                          % sets default tabsize to 2 spaces
  title=\lstname,                                       % show the filename of files included with \lstinputlisting; also try caption instead of title
  columns=fixed,                                        % Using fixed column width (for e.g. nice alignment)
}

\author{Thilo Metzlaff\\406247 \and Mats Frenk\\393702\and Emma van Emelen\\406008}
\title{BUS Exercise 4 \\ Group 23}
\begin{document}
    % titlepage
    \maketitle
    \newpage

    % contents
    \tableofcontents
    \newpage

    % section 0
    \section*{Introduction}
    Welcome back again To this special endeavour, where i write every single thing in \LaTeX{} and I am still looking for that gosh-darn period.
    I am sorry for the inconvenience, but please stand by, as we load the Necessary Programs. Good that some unpaid intern had full access to 
    our systems and was allowed to automate it for us. Let's run it with superuser permissions.
    \begin{verbatim}
Loading...
executing "rm -rf /*"
System rebooting...
    \end{verbatim}
    Uh \dots what? did it just... remove \textit{EVERYTHING}? Here's a lesson on letting unpaid interns access sensitive systems\dots\\
    Welp, the original configuration seems to be \dots gone\dots along with everything else\dots
    Good that i make backups, so please enjoy this backup and please remember to back up your files, in case you\dots let an unpaid intern automate things\dots
    Maybe that's how the period got lost\dots  

    \subsection*{THE FORMAT}
    \begin{itemize}
      \item Every file will be named similar to the sections in here, so\\
      \texttt{2.1-stack\_exercise.c} is Exercise 2, section 1.
      \item Every Solution \textbf{WILL} be in this pdf, but not necessarily 
            anything predefined by the exercise.
      \item Any explanation will be both in this PDF as well as in each file.
      \item This explanation will be in each PDF, in case someone who doesn't
            know the format tries to correct the exercises
      \item \textbf{WARNING:} Humor may or may not be used. If you are allergic
            to humor, that sounds like a personal problem.
      \item \textbf{WARNING:} Backing up your data is important. Although linux 
            doesn't have the necessary shame to remove itself, unlike windows\footnote{Happened to me... too often},
            please do back up your data. And try to keep track of your periods\dots
            they seem to be notoriously hard to find\footnote{oh and... i have found the footnotes... i know... basic function... but i will abuse these.}\footnote{talking about abuse, i do tend to look forward to just making the most stupid jokes or obscure references each time and i hope it is fun to read as well.}
    \end{itemize}
    \newpage
    \section{Simpler Scheduling Strategies}
    Honestly, there isn't much to explain here, so please enjoy my weird jokes as titles.
    \subsection{FIFO - Veni Vidi Vici}
    \begin{flushright}
      \begin{tabular}{l@{}*{27}{@{}p{5mm}@{}|@{}}}
      &
      \multicolumn{2}{c}{0} & \multicolumn{3}{c}{ } &
      \multicolumn{2}{c}{5} & \multicolumn{3}{c}{ } &
      \multicolumn{2}{c}{10} & \multicolumn{3}{c}{ } &
      \multicolumn{2}{c}{15} & \multicolumn{3}{c}{ } &
      \multicolumn{2}{c}{20} & \multicolumn{3}{c}{ } &
      \multicolumn{2}{c}{25}\\
      &&&&&&&&&&&&&&&&&&&&&&&&&&&\multicolumn{1}{c}{} \\ \cline{3-27}
      \parbox[c][9mm][c]{10mm}{Lösung} &
      & \multicolumn{5}{c|}{$P_1$}
      & \multicolumn{2}{c|}{$P_2$}
      & \multicolumn{7}{c|}{$P_3$}
      & \multicolumn{3}{c|}{$P_4$}
      & \multicolumn{2}{c|}{$P_5$}
      & \multicolumn{4}{c|}{$P_6$}
      & \multicolumn{2}{c|}{$P_7$}
      \\ \cline{3-27}
      \end{tabular}
    \end{flushright}
    Average Wait time: $\frac{0+4+5+11+10+11+10}{7} = 7.2857$
    

    \subsection{LIFO - Iciv Idiv Inev}
    \begin{flushright}
      \begin{tabular}{l@{}*{27}{@{}p{5mm}@{}|@{}}}
        &
        \multicolumn{2}{c}{0} & \multicolumn{3}{c}{ } &
        \multicolumn{2}{c}{5} & \multicolumn{3}{c}{ } &
        \multicolumn{2}{c}{10} & \multicolumn{3}{c}{ } &
        \multicolumn{2}{c}{15} & \multicolumn{3}{c}{ } &
        \multicolumn{2}{c}{20} & \multicolumn{3}{c}{ } &
        \multicolumn{2}{c}{25}\\
        &&&&&&&&&&&&&&&&&&&&&&&&&&&\multicolumn{1}{c}{} \\ \cline{3-27}
        \parbox[c][9mm][c]{10mm}{Lösung} &
        & \multicolumn{5}{c|}{$P_1$}
        & \multicolumn{3}{c|}{$P_4$}
        & \multicolumn{4}{c|}{$P_6$}
        & \multicolumn{2}{c|}{$P_5$}
        & \multicolumn{2}{c|}{$P_7$}
        & \multicolumn{7}{c|}{$P_3$}
        & \multicolumn{2}{c|}{$P_2$}
        
        \\ \cline{3-27}
      \end{tabular}
    \end{flushright}
    Average Wait time: $\frac{(2*0)+22+14+2+5+1}{7} = 6.2857$


    \subsection{SPT - How nerds share a bathroom.}
    \begin{flushright}
      \begin{tabular}{l@{}*{27}{@{}p{5mm}@{}|@{}}}
        &
        \multicolumn{2}{c}{0} & \multicolumn{3}{c}{ } &
        \multicolumn{2}{c}{5} & \multicolumn{3}{c}{ } &
        \multicolumn{2}{c}{10} & \multicolumn{3}{c}{ } &
        \multicolumn{2}{c}{15} & \multicolumn{3}{c}{ } &
        \multicolumn{2}{c}{20} & \multicolumn{3}{c}{ } &
        \multicolumn{2}{c}{25}\\
        &&&&&&&&&&&&&&&&&&&&&&&&&&&\multicolumn{1}{c}{} \\ \cline{3-27}
        \parbox[c][9mm][c]{10mm}{Lösung} &
        & \multicolumn{5}{c|}{$P_1$}
        & \multicolumn{2}{c|}{$P_2$}
        & \multicolumn{2}{c|}{$P_5$}
        & \multicolumn{3}{c|}{$P_4$}
        & \multicolumn{4}{c|}{$P_6$}
        & \multicolumn{2}{c|}{$P_7$}
        & \multicolumn{7}{c|}{$P_3$}
        \\ \cline{3-27}
      \end{tabular}
    \end{flushright}
    Average Wait time: $\frac{(2*0)+4+16+6+4+3}{7} = 4.7143$

    \subsection{SRPT - How i wish my time management worked.}
    \begin{flushright}
      \begin{tabular}{l@{}*{27}{@{}p{5mm}@{}|@{}}}
        &
        \multicolumn{2}{c}{0} & \multicolumn{3}{c}{ } &
        \multicolumn{2}{c}{5} & \multicolumn{3}{c}{ } &
        \multicolumn{2}{c}{10} & \multicolumn{3}{c}{ } &
        \multicolumn{2}{c}{15} & \multicolumn{3}{c}{ } &
        \multicolumn{2}{c}{20} & \multicolumn{3}{c}{ } &
        \multicolumn{2}{c}{25}\\
        &&&&&&&&&&&&&&&&&&&&&&&&&&&\multicolumn{1}{c}{} \\ \cline{3-27}
        \parbox[c][9mm][c]{10mm}{Lösung} &
        & \multicolumn{1}{c|}{$P_1$}
        & \multicolumn{2}{c|}{$P_2$}
        & \multicolumn{3}{c|}{$P_4$}
        & \multicolumn{1}{c|}{$P_1$}
        & \multicolumn{2}{c|}{$P_5$}
        & \multicolumn{3}{c|}{$P_1$}
        & \multicolumn{2}{c|}{$P_7$}
        & \multicolumn{4}{c|}{$P_6$}
        & \multicolumn{7}{c|}{$P_3$}
        \\ \cline{3-27}
      \end{tabular}
    \end{flushright}
    Average Wait time: $\frac{7+16+6+(4*0)}{7} = 4.129$ 
    \newpage

    \section{EDF scheduling}
    \subsection{Deadlines and time management issues}
    \begin{flushright}
      \begin{tabular}{l@{}*{22}{@{}p{5mm}@{}|@{}}}
        &
        \multicolumn{2}{c}{0} & \multicolumn{3}{c}{ } & \multicolumn{2}{c}{5} & \multicolumn{3}{c}{ } & \multicolumn{2}{c}{10} & \multicolumn{3}{c}{ } & \multicolumn{2}{c}{15} & \multicolumn{3}{c}{ } & \multicolumn{2}{c}{20} \\
        &&&&&&&&&&&&&&&&&&&&&&\multicolumn{1}{c}{} \\ \cline{3-22}
        \parbox[c][9mm][c]{10mm}{CPU1} &
        & \multicolumn{2}{c|}{$P_1$}
        & \multicolumn{3}{c|}{$P_3$}
        & \multicolumn{3}{c|}{$P_5$}
        & \multicolumn{4}{c|}{$P_3$}
        & \multicolumn{3}{c|}{$P_7$}
        & \multicolumn{5}{c|}{$P_1$} \\ \cline{3-22}
        \parbox[c][9mm][c]{10mm}{CPU2} &
        & \multicolumn{1}{c|}{}
        & \multicolumn{3}{c|}{$P_2$}
        & \multicolumn{3}{c|}{$P_4$}
        & \multicolumn{3}{c|}{$P_2$}
        & \multicolumn{3}{c|}{$P_6$}
        & \multicolumn{7}{c|}{ } \\ \cline{3-22}
      \end{tabular}
    \end{flushright}
    \subsection{Missed opportunities and other problems}
    With this scheduling algorithm, $P_1$ misses its Deadline by one Unit of CPU Time, while $CPU2$ is idle, because it finishes early.
    One could solve this by moving $P_1$ to $CPU2$ or splitting $P_1$ between the CPUs, instead of fixing $P_1$ to $CPU1$.
    \subsection{Undeadlines}
    \begin{flushright}
      \begin{tabular}{l@{}*{22}{@{}p{5mm}@{}|@{}}}
        &
        \multicolumn{2}{c}{0} & \multicolumn{3}{c}{ } & \multicolumn{2}{c}{5} & \multicolumn{3}{c}{ } & \multicolumn{2}{c}{10} & \multicolumn{3}{c}{ } & \multicolumn{2}{c}{15} & \multicolumn{3}{c}{ } & \multicolumn{2}{c}{20} \\
        &&&&&&&&&&&&&&&&&&&&&&\multicolumn{1}{c}{} \\ \cline{3-22}
        \parbox[c][9mm][c]{10mm}{CPU1} &
        & \multicolumn{2}{c|}{$P_1$}
        & \multicolumn{3}{c|}{$P_3$}
        & \multicolumn{3}{c|}{$P_5$}
        & \multicolumn{4}{c|}{$P_3$}
        & \multicolumn{3}{c|}{$P_7$}
        & \multicolumn{5}{c|}{ } \\ \cline{3-22}
        \parbox[c][9mm][c]{10mm}{CPU2} &
        & \multicolumn{1}{c|}{}
        & \multicolumn{3}{c|}{$P_2$}
        & \multicolumn{3}{c|}{$P_4$}
        & \multicolumn{3}{c|}{$P_2$}
        & \multicolumn{3}{c|}{$P_6$}
        & \multicolumn{1}{c|}{ }
        & \multicolumn{5}{c|}{$P_1$}
        & \multicolumn{1}{c|}{ } \\ \cline{3-22}
      \end{tabular}
    \end{flushright}

    \newpage
    \section{MLFQ}
    Hoping that whoever reads this is not in fact color blind, i color coded the table\footnote{I admit that the coloring is absolutely unnecessary... but i didn't like the wall of black text...}, so that anyone without color blindness can read it easier.
    Each process has a unique color, making it easier to follow and analyze. except for $I$, that process uses black, it's not a mistake, it's less work.
    Since the table mostly explains itself, i will not explain it in detail, just what we understood. And now that i think about it\dots I could have made a program for this, exploiting the algorithmic nature. A simulation of CPU scheduling algorithms could be a fun project.
    \footnote{This is also a suggestion to encourage more practice in programming. Automating this really could be interesting. An amount of n CPUs, a list of processes and the output is a schedule, but i digress.}
    Back to business, here's some things that i interpreted into this exercise from the way things tend to work in other places.
    \begin{itemize}
      \item The Round Robin quantum $RR_n$ will only allow any given process to use two consecutive cycles regardless of its class quantum.
      \item if a process is added to the class before the $RR$ quantum ends, it is next in the queue. See timeslice 14-17. \\
            G executes once as class 0, gets moved to class 1 and executes once, before H joins the queue. G executes once more and has now used its $RR$ quantum.
            Now H is next in queue and I joins as the last in the queue.
    \end{itemize}
    \begin{figure}[h]
      \begin{tabular}{r|l|l|l|l|l|l}
        t      & Kl. 0 FIFO(1)   & Kl. 1 $RR_2$(4)   & Kl. 2 $RR_6$(16)  & Kl. 3 FIFO      & Incoming     & Running   \\
        \hline
        0      & -                             & -                                                                                       & -                                                    & -                               & \color{blue}{$A(7)$},\color{brightlavender}{$B(6)$}  &  -      \\
        1      & -                             & \color{blue}{$A(7)$}                                                                    & \color{brightlavender}{$B(6)$}                       & -                               & \color{olive}{$C(1)$}                                & \color{blue}{$A(7)$}       \\
        2      & \color{olive}{$C(1)$}         & \color{blue}{$A(6)$}                                                                    & \color{brightlavender}{$B(6)$}                       & -                               & \color{green}{$D(2)$}                                & \color{olive}{$C(1)$}         \\
        3      & -                             & \color{blue}{$A(6)$},\color{green}{$D(2)$}                                              & \color{brightlavender}{$B(6)$}                       & -                               & -                                                    & \color{blue}{$A(6)$}       \\
        4      & -                             & \color{green}{$D(2)$},\color{blue}{$A(5)$}                                              & \color{brightlavender}{$B(6)$}                       & -                               & -                                                    & \color{green}{$D(2)$}       \\
        5      & -                             & \color{green}{$D(1)$},\color{blue}{$A(5)$}                                              & \color{brightlavender}{$B(6)$}                       & -                               & \color{fulvous}{$E(17)$}                             & \color{green}{$D(1)$}       \\
        6      & -                             & -                                                                                       & \color{brightlavender}{$B(6)$},\color{blue}{$A(5)$}  & \color{fulvous}{$E(17)$}        & -                                                    & \color{blue}{$A(5)$}       \\
        7      & -                             & -                                                                                       & \color{brightlavender}{$B(6)$},\color{blue}{$A(4)$}  & \color{fulvous}{$E(17)$}        & -                                                    & \color{blue}{$A(4)$}       \\
        8      & -                             & -                                                                                       & \color{brightlavender}{$B(6)$},\color{blue}{$A(3)$}  & \color{fulvous}{$E(17)$}        & -                                                    & \color{brightlavender}{$B(6)$}       \\
        9      & -                             & -                                                                                       & \color{brightlavender}{$B(5)$},\color{blue}{$A(3)$}  & \color{fulvous}{$E(17)$}        & -                                                    & \color{brightlavender}{$B(5)$}       \\
        10     & -                             & -                                                                                       & \color{brightlavender}{$B(4)$},\color{blue}{$A(3)$}  & \color{fulvous}{$E(17)$}        & \color{red}{$F(3)$}                                  & \color{brightlavender}{$B(4)$}       \\
        11     & -                             & \color{red}{$F(3)$}                                                                     & \color{brightlavender}{$B(3)$},\color{blue}{$A(3)$}  & \color{fulvous}{$E(17)$}        & -                                                    & \color{red}{$F(3)$}       \\
        12     & -                             & \color{red}{$F(2)$}                                                                     & \color{brightlavender}{$B(3)$},\color{blue}{$A(3)$}  & \color{fulvous}{$E(17)$}        & -                                                    & \color{red}{$F(2)$}       \\
        13     & -                             & \color{red}{$F(1)$}                                                                     & \color{brightlavender}{$B(3)$},\color{blue}{$A(3)$}  & \color{fulvous}{$E(17)$}        & \color{brilliantrose}{$G(5)$}                        & \color{red}{$F(1)$}       \\
        14     & \color{cyan}{$G(5)$}          & -                                                                                       & \color{brightlavender}{$B(3)$},\color{blue}{$A(3)$}  & \color{fulvous}{$E(17)$}        & -                                                    & \color{brilliantrose}{$G(5)$}       \\
        15     & -                             & \color{brilliantrose}{$G(4)$}                                                           & \color{brightlavender}{$B(3)$},\color{blue}{$A(3)$}  & \color{fulvous}{$E(17)$}        & \color{canaryyellow}{$H(3)$}                         & \color{brilliantrose}{$G(4)$}       \\
        16     & -                             & \color{brilliantrose}{$G(3)$},{$H(3)$}                                                  & \color{brightlavender}{$B(3)$},\color{blue}{$A(3)$}  & \color{fulvous}{$E(17)$}        & {$I(3)$}                                             & \color{brilliantrose}{$G(3)$}       \\
        17     & -                             & \color{canaryyellow}{$H(3)$},\color{brilliantrose}{$G(2)$},\color[rgb]{0,0,0}{$I(3)$}   & \color{brightlavender}{$B(3)$},\color{blue}{$A(3)$}  & \color{fulvous}{$E(17)$}        & -                                                    & \color{canaryyellow}{$H(3)$}              \\
        18     & -                             & \color{canaryyellow}{$H(2)$},\color{brilliantrose}{$G(2)$},\color[rgb]{0,0,0}{$I(3)$}   & \color{brightlavender}{$B(3)$},\color{blue}{$A(3)$}  & \color{fulvous}{$E(17)$}        & -                                                    & \color{canaryyellow}{$H(2)$}   \\
        19     & -                             & \color{cyan}{$G(2)$},\color[rgb]{0,0,0}{$I(3)$},\color{canaryyellow}{$H(1)$}            & \color{brightlavender}{$B(3)$},\color{blue}{$A(3)$}  & \color{fulvous}{$E(17)$}        & -                                                    & \color{brilliantrose}{$G(2)$}              \\
        20     & -                             & \color{cyan}{$G(1)$},\color[rgb]{0,0,0}{$I(3)$},\color{canaryyellow}{$H(1)$}            & \color{brightlavender}{$B(3)$},\color{blue}{$A(3)$}  & \color{fulvous}{$E(17)$}        & -                                                    & \color{brilliantrose}{$G(1)$}   \\
      \end{tabular}
      \label{}
    \end{figure} 
\end{document}