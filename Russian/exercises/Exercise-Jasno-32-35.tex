% general format definition
\documentclass[a4paper, 11pt]{article}
\usepackage[margin=.9in]{geometry}
\usepackage[utf8]{inputenc}
\usepackage[T1,T2A]{fontenc}
\usepackage[english,russian]{babel}

% extra packages
\usepackage{hyperref}
\usepackage{listings}
\usepackage{color}
\usepackage{amssymb}
\usepackage{amsmath}
\usepackage{mathtools}
\usepackage{microtype}
\usepackage{stmaryrd}
\usepackage{tikz}
\usepackage{booktabs}
\usepackage{stmaryrd}

% Syntax highlighting
\definecolor{commentsColor}{rgb}{0.497495, 0.497587, 0.497464}
\definecolor{keywordsColor}{rgb}{0.000000, 0.000000, 0.635294}
\definecolor{stringColor}{rgb}{0.558215, 0.000000, 0.135316}
\lstset{
  backgroundcolor=\color{white},                        % choose the background color; you must add \usepackage{color} or \usepackage{xcolor}
  basicstyle=\footnotesize,                             % the size of the fonts that are used for the code
  breakatwhitespace=false,                              % sets if automatic breaks should only happen at whitespace
  breaklines=true,                                      % sets automatic line breaking
  captionpos=b,                                         % sets the caption-position to bottom
  commentstyle=\color{commentsColor}\textit,            % comment style
  deletekeywords={},                                    % if you want to delete keywords from the given language
  escapeinside={\%*}{*)},                               % if you want to add LaTeX within your code
  extendedchars=true,                                   % lets you use non-ASCII characters; for 8-bits encodings only, does not work with UTF-8
  frame=tb,	                   	                        % adds a frame around the code
  keepspaces=true,                                      % keeps spaces in text, useful for keeping indentation of code (possibly needs columns=flexible)
  keywordstyle=\color{keywordsColor}\bfseries,          % keyword style
  otherkeywords={True,False,true,false,null,None,NULL}, % if you want to add more keywords to the set
  numbers=left,                                         % where to put the line-numbers; possible values are (none, left, right)
  numbersep=5pt,                                        % how far the line-numbers are from the code
  numberstyle=\tiny\color{commentsColor},               % the style that is used for the line-numbers
  rulecolor=\color{black},                              % if not set, the frame-color may be changed on line-breaks within not-black text (e.g. comments (green here))
  showspaces=false,                                     % show spaces everywhere adding particular underscores; it overrides 'showstringspaces'
  showstringspaces=false,                               % underline spaces within strings only
  showtabs=false,                                       % show tabs within strings adding particular underscores
  stepnumber=1,                                         % the step between two line-numbers. If it's 1, each line will be numbered
  stringstyle=\color{stringColor},                      % string literal style
  tabsize=2,	                                          % sets default tabsize to 2 spaces
  title=\lstname,                                       % show the filename of files included with \lstinputlisting; also try caption instead of title
  columns=fixed,                                        % Using fixed column width (for e.g. nice alignment)
}

\author{Emma van Emelen}
\title{Russian Homework\\\Large Jasno - Page 32-35}
\begin{document}
     \maketitle
     \newpage
     \tableofcontents
     \newpage

     \section{Section A}
     \subsection{a/b}
     \begin{itemize}
       \item сын сына $\rightarrow$ внук
       \item мать отца $\rightarrow$ бабушка
       \item родители $\rightarrow$ отец и мать
       \item отец отца $\rightarrow$ дедушка
       \item дочь сына $\rightarrow$ внучка
       \item дети папы и мамы $\rightarrow$ брат и сестра
     \end{itemize}
     \subsection{Exercise 1a}
     Im Russischen gibt es vier Zischlaute: \textbf{ж, ч, ш} und \textbf{щ}.
     \textbf{ж} wird \textbf{stimmhaft} gesprochen, die anderen zischlaute sind stimmlos.
     \subsection{Exercise 1b}
     жена, дочь, дочка, бабушка, наш, муж, ваша, внучка, дедушка, ещё, наши, жить, москвичка
     \subsection{Exercise 1c}
     \begin{itemize}
      \item[\textbf{ж:}] пожалуйста
      \item[\textbf{ч:}] очень, грамматические, четыре
      \item[\textbf{ш:}] нерушимость, груша
     \end{itemize}
     \subsection{Excercise 2}
     \begin{itemize}
      \item Здравствуйте, Николай Петрович!
      \item Здравствуйте, Виктор Сергеевич! Познакомьтесь, пожалуйста, вот моя семья. Это моя жена, Елена Михаиловна Цветаева.
      \item Очень приятно. Меня зовут Виктор Сергеевич, а моя фамилия - Кузнецов.
      \item А это наши дети. Вот это моя дочь, Алина. Она ещё студентка.
      \item Очень приятно. А это ваш сын?
      \item Да, конечно, это мой сын, Валери.
     \end{itemize}
     \newpage

     \subsection{Exercise 3}
     \begin{enumerate}
      \item Это мой дедушка. Его зовут Александр.
      \item А вот это наши друзья. Их зовут Таня и Пётр.
      \item Познакомьтесь, это моя внучка. Её зовут Аня.
      \item Это ваши дети? Как их зовут? 
      \item Посмотрите, это мой отец. Его зовут Артём.
      \item Это твоя подруга? Как её зовут?
     \end{enumerate}
     \section{Section B}
     \subsection{Excercise 1a}
     \begin{enumerate}
      \item Наталья живёт в 
      \item Она - учительница
      \item У неё есть муж. Его зовут Алексей.
      \item У Натальи один брат.
      \item Света - дочь Натальи.
      \item Родители Натальи живут на севере города.
     \end{enumerate}
     \subsection{Exercise 1b}
     \begin{itemize}
       \item 1 $\rightarrow$ g
       \item 2 $\rightarrow$ c
       \item 3 $\rightarrow$ f
       \item 4 $\rightarrow$ d
       \item 5 $\rightarrow$ b
       \item 6 $\rightarrow$ a
       \item 7 $\rightarrow$ e
       \item 8 $\rightarrow$ h
     \end{itemize}
     \subsection{Exercise 2}
     \begin{enumerate}
      \item У Антона есть брат.
      \item У Кати нет сестра.
      \item У Надежды есть дочь.
      \item У Михаила нет сын.
      \item У Лилии есть подруга.
      \item У Алины нет друг.
     \end{enumerate}
    \subsection{Exercise 3}
    \begin{enumerate}
      \item У тебя есть жена (муж)?
      \item У его есть дети? 
      \item У её есть семья?
      \item У Вас есть кошка?
      \item У их есть друзья?
    \end{enumerate}
    \subsection{Exercise 4}
    \begin{enumerate}
      \item Нет, у меня нет брат.
      \item Нет, у меня нет муж или жена.
      \item Нет, у меня нет сын.
      \item Нет, у меня нет дочь.
      \item Нет, у меня нет внук.
      \item Нет, у меня нет внучка.
      \item Да, его зовут Герберт.
      \item Да, его зовут Элизабет.
    \end{enumerate}
    \subsection{Exercise 5a}
    \begin{itemize}
      \item У Вадима есть один брат и одна сестры.
      \item У Любы есть три внуки и одна внучка.
      \item У Сергея есть четыре друга.
      \item У Анастасии есть два сына. 
      \item У Ивана есть одна дочь.
      \item У Наталии есть два друга.
    \end{itemize}
\end{document}